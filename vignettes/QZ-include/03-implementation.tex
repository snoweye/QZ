

\section[Implementation]{Implementation}
\label{sec:implementation}
\addcontentsline{toc}{section}{\thesection. Implementation}

Two main functions are \code{geigen()}
for generalized eigenvalues, and \code{qz()} for QZ decomposition with
reordering capability. Both functions are able to deal a single matrix $\bA$
or a paired matrices $(\bA,\bB)$ in both complex and real systems.
Both functions are wrapper functions for several lower level
\proglang{R} functions \code{qz.*()} which are also wrapper functions
via \code{.Call()}
for \proglang{C} and \proglang{Fortran} functions to LAPACK library 
version 3.4.2.


\subsection[R and LAPACK Functions]{R and LAPACK Functions}
\label{sec:lapack}
\addcontentsline{toc}{subsection}{\thesection. R and LAPACK Functions}

In the \pkg{QZ} package, \code{qz()}
will based on the data type of the input matrix $\bA$ or 
the paired matrices $(\bA,\bB)$ to select
accordingly the warpper functions \code{qz.*()}
as the feature of S3 method in \proglang{R}.

In \proglang{R}, one may still used the {qz.*()} individually if needed,
or convert the data type as desired. 
For example, one may use \code{as.complex()}
on the input real/double matrix $\bA$ or matrices $(\bA, \bB)$
to call the complex version of \code{qz()} as below
\begin{Code}[title=Use complex system via \code{as.complex()}]
set.seed(123)
A <- matrix(rnorm(9), nrow = 3)
B <- matrix(rnorm(9), nrow = 3)
ret.d <- qz(A, B, only.values = TRUE)
ret.c <- qz(matrix(as.complex(A), nrow(A)),
            matrix(as.complex(B), nrow(B)), only.values = TRUE)
str(ret.d)  # class of "dgges"
str(ret.c)  # class of "zgges"
\end{Code}

LAPACK library is internally incorporated in \pkg{QZ} including
complex*16 and double precision for complex and real systems, respectively.
To use internal LAPACK library of the \pkg{QZ},
one may add \code{--configure-args="--enable-iqz"}
to \code{R CMD INSTALL QZ_*.tar.gz}
when installing the \pkg{QZ} package.
\pkg{QZ} also allows some functions of
LAPACK and BLAS~\citep{BLAS}
independently to the \proglang{R}'s LAPACK and BLAS libraries
when some functions are not available.
Table~\ref{tab:qz_functions} provides a detail lists for the \code{qz.*()}
functions.

\begin{table}[h!tb]
\begin{center}
\caption{\pkg{QZ} functions}
\label{tab:qz_functions}
\begin{tabular}{c|l|c|l|l} \hline \hline
Function &
Wrapper          & Main Input        & System & Purpose \\
\hline

\multirow{2}{*}{\code{geigen()}} &
\code{qz.zgeev}  & $\bA$               & Complex & \multirow{2}{*}{Generalized eigenvalues} \\

&
\code{qz.dgeev}  & $\bA$               & Real    & \\
\hline

\multirow{4}{*}{\code{qz()}} &
\code{qz.zgees}  & $\bA$               & Complex & \multirow{2}{*}{QZ decomposition} \\

&
\code{qz.dgees}  & $\bA$               & Real    & \\
\cline{2-5}

&
\code{qz.ztrsen} & $\bT, \bQ$          & Complex & \multirow{2}{*}{Reordering} \\

&
\code{qz.dtrsen} & $\bT, \bQ$          & Real    & \\
\hline \hline

\multirow{2}{*}{\code{geigen()}} &
\code{qz.zggev}  & $(\bA,\bB)$         & Complex & \multirow{2}{*}{Generalized eigenvalues} \\

&
\code{qz.dggev}  & $(\bA,\bB)$         & Real    & \\
\hline

\multirow{4}{*}{\code{qz()}} &
\code{qz.zgges}  & $(\bA,\bB)$         & Complex & \multirow{2}{*}{QZ decomposition} \\

&
\code{qz.dgges}  & $(\bA,\bB)$         & Real    & \\
\cline{2-5}

&
\code{qz.ztgsen} & $(\bS,\bT),\bQ,\bZ$ & Complex & \multirow{2}{*}{Reordering} \\

&
\code{qz.dtgsen} & $(\bS,\bT),\bQ,\bZ$ & Real    & \\

\hline \hline
\end{tabular}
\end{center}
\end{table}


\subsection[Reordering]{Reording}
\label{sec:reordering}
\addcontentsline{toc}{subsection}{\thesection. Reording}

An extral MATLAB-like function \code{ordqz()} is also available to reordering
generalized eigenvalues and QZ decomposition results. The function which is
the combinations of \code{qz()} and \code{qz.ztrsen()/qz.dtrsen()} for
specified ordering keywords in Table~\ref{tab:ordqz}.
\begin{table}[h!tb]
\begin{center}
\caption{The \code{ordez()} keyword for reording.}
\label{tab:ordqz}
\begin{tabular}{c|l} \hline \hline
\code{keyword} & Purpose \\ \hline
\code{lhp} & Left-half (real(E) < 0) \\
\code{rhp} & Right-half (real(E) >= 0) \\
\code{udi} & Interior of unit disk (abs(E) < 1) \\
\code{udo} & Exterior of unit disk (abs(E) >= 1) \\
\code{ref} & Real eigenvalues first (top-left conner) \\
\code{cef} & Complex eigenvalues first (top-left conner) \\
\code{lhp.fo} & Left-half (real(E) < 0) and finite only \\
\code{rhp.fo} & Right-half (real(E) >= 0) and finite only \\
\code{udi.fo} & Interior of unit disk (abs(E) < 1) and finite only \\
\code{udo.fo} & Exterior of unit disk (abs(E) >= 1) and finite only \\
\hline\hline
\end{tabular}
\end{center}
\end{table}
The keywords \code{lhp}, \code{rhp}, \code{udi}, and \code{udo} are
implemented as (or similar to) the way Matlab does.
Additionally,
the keywords \code{*.fo} are implemented to select finite (generalized)
eigen values only.
Note that \code{select} argument of \code{qz()} allows users to specify
any order to group and reorder the decompositions.


\subsection[For Matlab Users]{For Matlab Users}
\label{sec:matlabe}
\addcontentsline{toc}{subsection}{\thesection. For matlab Users}

In Matlab,
one may need to specify options for complex or real systems
be used for obtaining the (generalized) eigenvalues and
for constructing the QZ decomposition.
Some Matlab versions may assume a ``complex'' system as the default
regardless input types.

Some Matlab versions may need additional options to turn on
``double precision'' for the QZ decomposition.
The error of the QZ decomposition
(in terms of maximum modulus of whole matrix)
may be around in the range of \code{1e-05} to \code{1e-06}
where the modulus is $r = Mod(z) = sqrt(x^2 + y^2)$.
Note that \code{1e-15} to \code{1e-16} should be expected if
double precission is used correctly.

The notations used in Matlab also may not be followed or consistent
as used in
\begin{itemize}
\item \pkg{QZ} in \proglang{R},
\item its depending LAPACK functions,
\item general QZ notations in some textbooks, or
\item Wikipedia webpage.
\end{itemize}
In newer Matlab, the documents say that
\code{[AA, BB, Q, Z] = qz(A, B)}
produces upper quasitriangular matrices \code{AA} and \code{BB},
and unitary matrices \code{Q} and \code{Z} such that
\code{Q * A * Z = AA} and \code{Q * B * Z = BB}.
Here, the \code{Z} in the Matlab notations
are the conjugated versions of the QZ decomposition
where $A = Q \times S \times Z^{*}$ and $B = Q \times T \times Z^{*}$.
i.e. The \code{Z} in Matlab is actually $Z^{*}$ in the general QZ notations.
The \code{AA} and \code{BB} in Matlab are the same $S$ and $T$ in the
general QZ notations.
